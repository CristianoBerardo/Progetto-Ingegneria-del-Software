\section{Requisiti non funzionali}

\begin{rnfenum}
    \item \textbf{Sicurezza}:
        
        \begin{rnfenum}
            \item Il sistema deve poter fornire accesso all'area privata di ogni singolo utente grazie a provider OAuth.
            \item Il sistema deve poter fornire accesso all'area privata di ogni singolo utente grazie a mail e password.
            \item Il sistema deve garantire la riservatezza delle password attraverso l'implementazione di meccanismi di sicurezza conformi alle migliori pratiche di settore e allo Stato dell'Arte tecnologico.
            \item Il sistema si dovrà appoggiare a provider esterni per garantire la sicurezza delle transazioni di denaro elettroniche.
        \end{rnfenum}
    
    \item \textbf{Prestazioni}: 

        \begin{rnfenum}
            \item Il sistema deve poter creare un report \textbf{RFXXX}, da quando la richiesta arriva al server, in meno di 3s.
            \item Il sistema deve essere in grado di rispondere entro massimo 3s a tutte le interazioni utente-sistema.
            \item Il sistema deve poter essere monitorato in qualsiasi momento, questo per verificare lo stato di tutto l'applicativo.
        \end{rnfenum}
    
    \item \textbf{Scalabilità}: 
        \begin{rnfenum}
            \item Il sistema deve essere in grado di poter scalare autonomamente in base al carico utente grazie a tecnologie quali Kubernetes e Container.
            \item Il sistema deve essere in grado di poter anche 100 richeste assieme senza perdere prestazione, come descritto da \textbf{RNF2.2}
        \end{rnfenum}

    \item \textbf{Affidabilità}:
        \begin{rnfenum}
            \item Il sistema deve essere avere un uptime del 99.9\%.
        \end{rnfenum}

    \item \textbf{Compatibilità}:
        \begin{rnfenum}
            \item Il sistema deve essere compatibile con i principali browser web Chrome, Firefox e Safari
            \item Il sistema deve essere reso disponibile per tutte le piattaforme: desktop, tablet e mobile
        \end{rnfenum}       

    \item \textbf{Normative}:
        \begin{rnfenum}
            \item Il sistema deve cancellare tutti i dati dell'utente in base alla normativa vigente italiana e secondo le leggi che regolano la privacy, GDPR.
        \end{rnfenum}

    \item \textbf{Sviluppo software}:
        \begin{rnfenum}
            \item Il sistema verrà implementato utilizzando TypeScript, il controllo sui tipi lo rende più immune ad errori umani.
            \item L'integrazione con la repository GitHub necessaria per mantenere lo storico e creare nuove features in parallelo e senza creare danni all'applicazione principale.
            \item Utilizzo di database documentale, MongoDB, per mantenere salvati i dati principali dell' applicazione.
            \item Utilizzo del database Firebase per garantire la possibilità di accesso con GoogleOAuth.
        \end{rnfenum}
    
    
\end{rnfenum}

\newpage
I requisiti non funzionali definiscono \textbf{COME} il sistema deve operare, specificando attributi di qualità e vincoli di sistema.\\
Caratteristiche:

\begin{itemize}
    \item Descrivono qualità e vincoli del sistema
    \item Spesso misurabili su una scala (velocità, affidabilità, ecc.)
    \item Impattano l'architettura e il design del sistema
\end{itemize}

Esempi di requisiti non funzionali:

\begin{enumerate}
    \item Prestazioni: Il sistema deve rispondere alle richieste entro 2 secondi
    \item Sicurezza: Le password devono essere criptate con algoritmo SHA-256
    \item Usabilità: L'interfaccia deve essere accessibile secondo standard WCAG 2.1
    \item Scalabilità: Il sistema deve supportare fino a 10.000 utenti concorrenti
    \item Affidabilità: Il sistema deve avere un uptime del 99,9%
    \item Manutenibilità: Il codice deve seguire standard di codifica specifici
    \item Compatibilità: L'applicazione deve funzionare sui browser Chrome, Firefox e Safari
\end{enumerate}

Differenze Principali

Focus:

\begin{itemize}
    \item Funzionali: cosa fa il sistema
    \item Non funzionali: come lo fa e quanto bene lo fa
\end{itemize}


Misurazione:

\begin{itemize}
    \item Funzionali: binaria (soddisfatto o non soddisfatto)
    \item Non funzionali: continua (livelli di prestazione)
\end{itemize}


Impatto sul design:

\begin{itemize}
    \item Funzionali: influenzano principalmente le funzionalità specifiche
    \item Non funzionali: influenzano l'intera architettura del sistema
\end{itemize}


Priorità:

\begin{itemize}
    \item Funzionali: facilmente prioritizzabili dai clienti
    \item Non funzionali: spesso determinati da considerazioni tecniche
\end{itemize}