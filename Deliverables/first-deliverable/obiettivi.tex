\newpage
\section{Obiettivi del progetto}

\paragraph{Cliente}


\begin{objitem}
    \item \textbf{Facilitare l'accesso ai prodotti locali} – Permettere ai clienti di prenotare comodamente da casa, evitando il rischio di trovare prodotti esauriti.
    \item  \textbf{Migliorare l'esperienza d'acquisto} – Eliminare le attese e garantire ai clienti un servizio più rapido e organizzato.
    \item  \textbf{Promuovere la filiera corta} – Avvicinare produttori e consumatori, riducendo l'intermediazione e garantendo prezzi più equi.
    \item  \textbf{Automatizzare la gestione degli ordini} – Inviare notifiche e reminder per semplificare il ritiro e ridurre le dimenticanze e gli sprechi.
    \item  \textbf{Creare un'esperienza digitale semplice e intuitiva} – Offrire un'interfaccia user-friendly per tutte le fasce d'età, in modo da coinvolgere anche le persone più giovani a consumare prodotti a chilometro zero.
    \item  \textbf{Garantire maggiore trasparenza sulla provenienza} – Dare informazioni dettagliate su origine, metodo di coltivazione e certificazioni dei prodotti.
    \item  \textbf{Fidelizzare i clienti con offerte e programmi fedeltà} – Premiare chi acquista regolarmente prodotti locali.
    \item \textbf{Cliente e Produttore, un legame forte} - Fornire al cliente la possibilità di esprimere il proprio gradimento dell'acquisto effettuato.
\end{objitem}

\paragraph{Venditore}
\begin{objitem}
    \item  \textbf{Rafforzare l'economia locale} – Sostenere i piccoli produttori, permettendo di raggiungere un maggior numero di clienti.
    \item  \textbf{Supportare i produttori locali} – Offrire ai venditori uno strumento digitale per gestire gli ordini, organizzare al meglio il lavoro e aumentare le vendite.
    \item  \textbf{Integrare sistemi di pagamento digitali} – Consentire pagamenti anticipati o alla consegna per una maggiore flessibilità.
    \item  \textbf{Creare un network tra produttori} – Offrire uno spazio per la collaborazione tra produttori, magari per promozioni congiunte.
    \item \textbf{Statistiche più accurate} - Fornire un report dettagliato degli incassi e della merce venduta
\end{objitem}

\newpage

\paragraph{Sostenibilità ambientale}    
\begin{objitem}
    \item  \textbf{Ridurre l'impatto ambientale} – Evitare lunghi trasporti e importazioni di prodotti, e favorire gli spostamenti sostenibili ed ecologici, abbassando le emissioni di $CO_2$.
    \item  \textbf{Incentivare il consumo di prodotti locali, freschi e stagionali} – Favorire un'alimentazione più sana e sostenibile, incentivando pratiche agricole a basso impatto.
    \item  \textbf{Educare al consumo consapevole} – Fornire informazioni su stagionalità, valori nutrizionali e ricette con i prodotti acquistati.
    \item  \textbf{Ridurre l'uso di imballaggi inutili} – Permettere ai clienti di prenotare e ritirare con i propri contenitori, diminuendo i rifiuti.
    \item \textbf{Riduzione dello spreco} – Possibilità per il venditore di aggiungere un insieme di prodotti scontati, in caso di scadenza a breve. 
\end{objitem}


