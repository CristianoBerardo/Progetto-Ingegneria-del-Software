\section{Requisiti funzionali per attore}


\textbf{RF1}: <id> the system has to <function>

\textbf{RF2}: <id> the system has to <function>

Ho creato un nuovo ambiente al posto di scrivere sempre RFnumero usa \verb|begin{rfenum}|:

\begin{rfenum}

    \item prova
    \item prova 2

\end{rfenum}

\paragraph{Reminder}
I requisiti funzionali descrivono \textbf{COSA} deve fare un'applicazione, specificando le funzionalità che il sistema deve fornire.

Caratteristiche:

\begin{itemize}
    \item Definiscono comportamenti specifici
    \item Sono misurabili in termini di completamento (funziona/non funziona)
    \item Descrivono interazioni tra il sistema e l'ambiente
\end{itemize}

Esempi di requisiti funzionali:

\begin{enumerate}
    \item Un utente deve poter registrare un account con email e password
    \item Il sistema deve inviare email di conferma quando un ordine viene completato
    \item Gli amministratori devono poter visualizzare tutti gli utenti registrati
    \item Il sistema deve permettere agli utenti di caricare file di dimensione massima 10MB
    \item Un utente deve poter effettuare una ricerca per parole chiave
\end{enumerate}