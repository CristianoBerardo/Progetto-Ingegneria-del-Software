\section{Requisiti Funzionali}

\begin{rfenum}
    \item \textbf{Utente}:
        \begin{rfenum}
            \item : Il sistema deve consentire all'utente di registrarsi tramite email e password.
            \item : Il sistema deve consentire all'utente di autenticarsi tramite email e password.
            \item : Il sistema deve permettere all'utente di selezionare un produttore/venditore dalla lista per accedere al relativo catalogo di prodotti.
            \item : Il sistema deve consentire all'utente di visualizzare in dettaglio il catalogo dei prodotti del produttore selezionato, includendo immagini, descrizione, prezzo e disponibilità.
            \item : Il sistema deve permettere all'utente di scegliere i prodotti desiderati e specificare le quantità da aggiungere al carrello.          
            \item : Il sistema deve offrire la possibilità di selezionare il metodo di pagamento preferito (online oppure presso il punto vendita) prima del checkout.
            \item : Il sistema deve gestire il processo di checkout in modo integrato e generando un riepilogo dettagliato dell'ordine.
            \item : Il sistema deve inviare una conferma d'ordine comprensiva dei dettagli dell'ordine, del metodo di pagamento scelto e delle informazioni per il ritiro.
            \item : Il sistema deve permettere all'utente di accedere a una sezione “I miei ordini” in cui visualizzare la cronologia degli ordini effettuati, con dettagli sullo stato e sui prodotti ordinati.

        \end{rfenum}
        
    \item \textbf{Produttore/Venditore}:
        \begin{rfenum}
            \item : Il sistema deve consentire ai produttori/venditori di registrarsi con email e password.
            \item : Il sistema deve consentire ai produttori/venditori di accedere con email e password.
            \item : Il sistema deve permettere ai produttori/venditori di inserire un nuovo prodotto con immagine, descrizione e quantità.
            \item: Il sistema deve permettere ai venditori/produttori di rimuovere un prodotto dalla lista.
            \item: Il sistema deve mostrare ai venditori/produttori i prodotti aggiunti precedentemente.
            \item : Il sistema deve notificare ai produttori/venditori la ricezione degli ordini.
            \item : Il sistema deve supportare transazioni digitali sicure per pagamenti anticipati.
        \end{rfenum}
        
    \item \textbf{Amministratore}:
        \begin{rfenum}
            \item : Il sistema deve permettere all'amministratore di visualizzare, modificare e gestire tutti gli account registrati.
            \item : Il sistema deve fornire strumenti per il controllo e il monitoraggio delle transazioni, degli ordini e delle prestazioni complessive.
            \item : Il sistema deve includere funzionalità per la segnalazione e la risoluzione di eventuali problemi o anomalie riscontrate dagli utenti.
        \end{rfenum}
\end{rfenum}

\newpage
I requisiti funzionali definiscono \textbf{COSA} deve fare il sistema, specificando le funzionalità che l'applicazione deve fornire.\\
Caratteristiche:
\begin{itemize}
    \item Definiscono comportamenti specifici
    \item Sono misurabili in termini di completamento (funziona/non funziona)
    \item Descrivono interazioni tra il sistema e l'ambiente
\end{itemize}

Esempi di requisiti funzionali:
\begin{enumerate}
    \item Un utente deve poter registrare un account con email e password.
    \item Il sistema deve inviare email di conferma quando un ordine viene completato.
    \item Gli amministratori devono poter visualizzare tutti gli utenti registrati.
    \item Il sistema deve permettere agli utenti di caricare file di dimensione massima 10MB.
    \item Un utente deve poter effettuare una ricerca per parole chiave.
\end{enumerate}
