\documentclass{scrartcl}
\usepackage[a4paper,top=2cm,bottom=2.5cm,left=2.5cm,right=2.5cm,marginparwidth=0cm]{geometry}
\usepackage[english]{babel}
\usepackage[linguistics]{forest}

\usepackage[
    doi=true,
    language=english,
    natbib=true,
    sortcites,
    style=unified,
    useprefix=true,
    ]{biblatex}


% Fonts, languages
\usepackage[warnings-off={mathtools-colon, mathtools-overbracket}]{unicode-math}
\usepackage{fontspec}
\defaultfontfeatures{Ligatures=TeX}
\usepackage[sb]{libertinus-otf}
\usepackage[scale=MatchLowercase]{FiraMono}
\usepackage{fontawesome5}

% Nicer tables
\usepackage{booktabs}
\usepackage[table]{xcolor}
\usepackage{colortbl}
\usepackage[nopatch=footnote]{microtype}
\usepackage{tabularx}

\usepackage[unicode,hidelinks]{hyperref}
\usepackage[normalem]{ulem} % for strikethrough \sout

\usepackage{scrlayer-scrpage}
\newpairofpagestyles{titlepage}{%
    \setkomafont{pageheadfoot}{%
        \small\normalfont 
    }
    \rohead{}
    \lohead{}
}
\pagestyle{scrheadings}
\setkomafont{pagenumber}{\footnotesize}%\sffamily}
\setkomafont{pageheadfoot}%
    {\small\addfontfeature{LetterSpace=10,Numbers=OldStyle}\scshape\sffamily}
\clearpairofpagestyles{}
\cfoot[\pagemark]{\pagemark}
% \rohead{\MakeLowercase{\emph{\shorttitle}}}
%\lohead{\MakeLowercase{\authorlast}}


% Titlepage
\setkomafont{author}{\large\sffamily}


% Chapter/section headings
\setcounter{secnumdepth}{3}

% Linguistics
\usepackage{langsci-gb4e}
\newcommand{\judgement}[1]{\makebox[0pt][r]{#1}}



\usepackage{cleveref}
\usepackage{enumitem}

%% Line breaks
\widowpenalty=10000
\clubpenalty=10000

% useful 
\usepackage{orcidlink}
\newcommand{\email}[1]{\href{mailto:#1}{#1}}

% copyright notice at end
\usepackage[framemethod=TikZ]{mdframed}
\newenvironment{ccnotice}[1]{%
    \begin{mdframed}[%
        linewidth=0pt,
        leftmargin=1pt,
        rightmargin=1pt,
        backgroundcolor=gray!10!white,
        font=\sffamily,
    ]\relax%
}{\end{mdframed}}


\usepackage[autostyle=true,english=american]{csquotes}
\renewcommand*{\mkccitation}[1]{ (#1)}

\usepackage{xcolor}
\usepackage{titlesec}
\definecolor{mycolor}{HTML}{2f5496}
\titleformat{\section}
              {\normalfont\Large\bfseries\color{mycolor}} % Formatting
              {\thesection}            % Label
              {2em}                   % Space after label
              {}  
\titleformat{\paragraph}
  {\normalfont\normalsize\bfseries}{\theparagraph}{1em}{}


% title page
\title{\fulltitle}
\date{}


\urlstyle{same}

\hypersetup{
    colorlinks=false,
    linkcolor=blue,
    filecolor=magenta,      
    urlcolor=cyan,
    pdftitle={Deliverable-D1-Gruppo3},
    pdfpagemode=FullScreen,
    }
\newcommand{\fulltitle}{\color{mycolor}Deriverable D1}

\newenvironment{objitem}{
    \begin{enumerate}
    \renewcommand{\labelenumi}{\textbf{O\arabic{enumi}}}
}
{\end{enumerate}}

\newenvironment{rfenum}{
    \begin{enumerate}
    \renewcommand{\labelenumi}{\textbf{RF\arabic{enumi}}}    
    \renewcommand{\labelenumii}{RF\arabic{enumi}.\arabic{enumii}}
    \renewcommand{\labelenumiii}{RF\arabic{enumi}.\arabic{enumii}.\arabic{enumiii}}
    
}
{\end{enumerate}}

\newenvironment{rnfenum}{
    \begin{enumerate}
    \renewcommand{\labelenumi}{\textbf{RNF\arabic{enumi}}}    
    \renewcommand{\labelenumii}{RNF\arabic{enumi}.\arabic{enumii}}
    \renewcommand{\labelenumiii}{RNF\arabic{enumi}.\arabic{enumii}.\arabic{enumiii}}

}
{\end{enumerate}}

\author{%
    \huge\color{mycolor}\textbf{Gruppo 3} \\\\
    Berardo Cristiano - 234428\\
    De Piccoli Martina - 235165\\
    Vettore Giacomo - 240396
}

% --------------- Start Document  -----------------------------

\begin{document}

\maketitle
\thispagestyle{titlepage}

\begin{abstract}
\noindent\textbf{Idea di progetto:} \textit{Sviluppare una piattaforma web che faciliti la prenotazione e la vendita di frutta e verdura del mercato contadino di Trento, incentivando il consumo di prodotti stagionali locali, favorendo pratiche sostenibili e nel rispetto del nostro territorio.}



% I venditori possono gestire ordini, pagamenti e scorte in modo efficiente.

% La piattaforma promuove la filiera corta, la trasparenza sulla provenienza dei prodotti e pratiche sostenibili come la riduzione degli imballaggi e l'ottimizzazione dei trasporti. Include funzionalità per fidelizzare i clienti, creare collaborazioni tra produttori e sensibilizzare sul consumo consapevole, sostenendo l'economia locale e riducendo l'impatto ambientale.


\end{abstract}

\section{Obiettivi}

\textbf{Utente}
\begin{itemize}
    \item \textbf{Facilitare l'accesso ai prodotti locali} – Permettere ai clienti di prenotare comodamente da casa, evitando il rischio di trovare prodotti esauriti.
    \item  \textbf{Migliorare l'esperienza d'acquisto} – Eliminare le attese e garantire ai clienti un servizio più rapido e organizzato.
    \item  \textbf{Promuovere la filiera corta} – Avvicinare produttori e consumatori, riducendo l'intermediazione e garantendo prezzi più equi.
    \item  \textbf{Automatizzare la gestione degli ordini} – Inviare notifiche e reminder per semplificare il ritiro e ridurre le dimenticanze e gli sprechi.
    \item  \textbf{Creare un'esperienza digitale semplice e intuitiva} – Offrire un'interfaccia user-friendly per tutte le fasce d'età.
    \item  \textbf{Garantire maggiore trasparenza sulla provenienza} – Dare informazioni dettagliate su origine, metodo di coltivazione e certificazioni dei prodotti.
    \item  \textbf{Fidelizzare i clienti con offerte e programmi fedeltà} – Premiare chi acquista regolarmente prodotti locali.
\end{itemize}
    
\textbf{Venditore}
\begin{itemize}
    \item  \textbf{Rafforzare l'economia locale} – Sostenere i piccoli produttori e artigiani, permettendo di raggiungere un maggior numero di clienti.
    \item  \textbf{Supportare i produttori locali} – Offrire ai venditori uno strumento digitale per gestire gli ordini, organizzare meglio il lavoro e aumentare le vendite.
    \item  \textbf{Integrare sistemi di pagamento digitali} – Consentire pagamenti anticipati o alla consegna per una maggiore flessibilità.
    \item  \textbf{Creare un network tra produttori} – Offrire uno spazio per la collaborazione tra contadini, magari per promozioni congiunte.
\end{itemize}


\textbf{Sostenibilità ambientale}    
\begin{itemize}
    \item  \textbf{Ridurre l'impatto ambientale} – Evitare lunghi trasporti e importazioni di prodotti, e favorire gli spostamenti sostenibili ed ecologici, abbassando le emissioni di CO₂.
    \item  \textbf{Incentivare il consumo di prodotti locali, freschi e stagionali} – Favorire un'alimentazione più sana e sostenibile, incentivando pratiche agricole a basso impatto.
    \item  \textbf{Educare al consumo consapevole} – Fornire informazioni su stagionalità, valori nutrizionali e ricette con i prodotti acquistati.
    \item  \textbf{Ridurre l'uso di imballaggi inutili} – Permettere ai clienti di prenotare e ritirare con i propri contenitori, diminuendo i rifiuti.
\end{itemize}




\newpage
\section{Attori del sistema}
\vspace{-8mm}
\begin{figure}[!h]
    \centering
    \includegraphics[width=1\linewidth]{Deliverables/first-deliverable/img/mappa_attori.png}
    \label{fig:mindmap}
\end{figure}

\vspace{-10mm}

\begin{attori}
    \item \textbf{Utente}: Generico utente che ha la possibilità di visualizzare la lista delle aziende produttrici, la loro valutazione e le relative recensioni, di consultare il catalogo dei prodotti in vendita, di ricercare aziende e prodotti tramite parole chiave e di visualizzare le ultime notizie ed aggiornamenti riguardanti il mercato contadino di Trento, in aggiunta ad orari e luoghi di svolgimento dello stesso.
    \begin{enumerate}
        \item \textbf{Utente non autenticato}: Interazioni limitate come Utente. Ha inoltre la possibilità di registrarsi (Sign up), di autenticarsi (Log in) tramite email e password o di recuperare quest'ultima.
        
        \item \textbf{Utente autenticato}: Utente che ha fatto l'accesso al sistema e che ha in aggiunta la possibilità di accedere ad un'area privata dove può modificare i propri dati.
    
        \begin{enumerate}
            \item \textbf{Cliente}: Dispone di tutte le interazioni possibili per l'utente autenticato. In aggiunta può prenotare i prodotti in vendita, richiedere la consegna a domicilio, e scegliere fra due opzioni di pagamento: online o presso il punto vendita. L'area privata consentirà di visualizzare le prenotazioni effettuate e i dettagli relativi all'account. Ha inoltre la possibilità di visualizzare e lasciare recensioni ai produttori. 
            
            \item \textbf{Produttore}: Dispone di tutte le interazioni possibili per l'utente autenticato. In aggiunta ha la possibilità di gestire i prodotti in vendita, inserendo i prodotti nuovi con relativi dettagli, e rimuovendo quelli esauriti. Può inoltre inserire offerte e promozioni per i clienti. L'area privata consentirà di monitorare sia gli ordini attivi che quelli completati, anche mediante l'utilizzo di dashborad interattive e la generazione di report relativi alle vendite e incassi.
            
             \item \textbf{Amministratore}: Ha la possibilità di visualizzare, controllare e modificare tutti gli account registrati in modo da risolvere eventuali problematiche. Monitora l’andamento della piattaforma, supervisiona il corretto utilizzo della stessa e modera le recensioni e feedback degli utenti.
             Ha infine il controllo completo sulla gestione degli avvisi pubblici: può pubblicare aggiornamenti su spostamenti del mercato, chiusure straordinarie ed eventi speciali.

        \end{enumerate}
    \end{enumerate}

    \item \textbf{Servizi Interni}
    \begin{enumerate}
        \item \textbf{Database}: Memorizza dati su utenti, venditori e prodotti, gestisce ordini, pagamenti e recensioni. Conserva inoltre dati su disponibilità e stagionalità dei prodotti e archivia statistiche e report di vendita.
        \item \textbf{Servizio di Pagamento}: Gestisce le transazioni online in sicurezza e supporta pagamenti flessibili (anticipati o alla consegna).
        \item \textbf{Servizio di Autenticazioane}: Gestisce la registrazione e il login degli utenti. Supporta autenticazione tramite email e password. Controlla permessi e ruoli (cliente, venditore e amministratore) e garantisce la sicurezza dei dati.
        \item \textbf{Servizio di Notificazione}: Invia conferme d’ordine ai clienti, ricorda le scadenze per il ritiro degli ordini e notifica i venditori sugli ordini ricevuti. 
    \end{enumerate}
\end{attori}

\newpage
\section{Requisiti Funzionali}

\begin{rfenum}
    \item \textbf{Utente}:
        \begin{rfenum}
            \item : Il sistema deve consentire all'utente di registrarsi tramite email e password.
            \item : Il sistema deve consentire all'utente di autenticarsi tramite email e password.
            \item : Il sistema deve permettere all'utente di selezionare un produttore/venditore dalla lista per accedere al relativo catalogo di prodotti.
            \item : Il sistema deve consentire all'utente di visualizzare in dettaglio il catalogo dei prodotti del produttore selezionato, includendo immagini, descrizione, prezzo e disponibilità.
            \item : Il sistema deve permettere all'utente di scegliere i prodotti desiderati e specificare le quantità da aggiungere al carrello.          
            \item : Il sistema deve offrire la possibilità di selezionare il metodo di pagamento preferito (online oppure presso il punto vendita) prima del checkout.
            \item : Il sistema deve gestire il processo di checkout in modo integrato e generando un riepilogo dettagliato dell'ordine.
            \item : Il sistema deve inviare una conferma d'ordine comprensiva dei dettagli dell'ordine, del metodo di pagamento scelto e delle informazioni per il ritiro.
            \item : Il sistema deve permettere all'utente di accedere a una sezione “I miei ordini” in cui visualizzare la cronologia degli ordini effettuati, con dettagli sullo stato e sui prodotti ordinati.

        \end{rfenum}
        
    \item \textbf{Produttore/Venditore}:
        \begin{rfenum}
            \item : Il sistema deve consentire ai produttori/venditori di registrarsi con email e password.
            \item : Il sistema deve consentire ai produttori/venditori di accedere con email e password.
            \item : Il sistema deve permettere ai produttori/venditori di inserire un nuovo prodotto con immagine, descrizione e quantità.
            \item: Il sistema deve permettere ai venditori/produttori di rimuovere un prodotto dalla lista.
            \item: Il sistema deve mostrare ai venditori/produttori i prodotti aggiunti precedentemente.
            \item : Il sistema deve notificare ai produttori/venditori la ricezione degli ordini.
            \item : Il sistema deve supportare transazioni digitali sicure per pagamenti anticipati.
        \end{rfenum}
        
    \item \textbf{Amministratore}:
        \begin{rfenum}
            \item : Il sistema deve permettere all'amministratore di visualizzare, modificare e gestire tutti gli account registrati.
            \item : Il sistema deve fornire strumenti per il controllo e il monitoraggio delle transazioni, degli ordini e delle prestazioni complessive.
            \item : Il sistema deve includere funzionalità per la segnalazione e la risoluzione di eventuali problemi o anomalie riscontrate dagli utenti.
        \end{rfenum}
\end{rfenum}

\newpage
I requisiti funzionali definiscono \textbf{COSA} deve fare il sistema, specificando le funzionalità che l'applicazione deve fornire.\\
Caratteristiche:
\begin{itemize}
    \item Definiscono comportamenti specifici
    \item Sono misurabili in termini di completamento (funziona/non funziona)
    \item Descrivono interazioni tra il sistema e l'ambiente
\end{itemize}

Esempi di requisiti funzionali:
\begin{enumerate}
    \item Un utente deve poter registrare un account con email e password.
    \item Il sistema deve inviare email di conferma quando un ordine viene completato.
    \item Gli amministratori devono poter visualizzare tutti gli utenti registrati.
    \item Il sistema deve permettere agli utenti di caricare file di dimensione massima 10MB.
    \item Un utente deve poter effettuare una ricerca per parole chiave.
\end{enumerate}

\section{Requisiti non Funzionali}

\paragraph{cristiano}

\newpage
\section{Use case diagrams e casi d'uso}

\begin{figure}[h!]
    \centerline{\includesvg[width=1\linewidth]{img/UseCase/UseCase-Diagram.svg}}
    \caption{Use Case diagram}
    \label{fig:useCase}
\end{figure}

\newpage
\subsection{Use case table $-$ Utente non autenticato e Cliente}

\begin{figure}[h!]
    \centering
    \includegraphics[trim= 1.8cm 4.3cm 5cm 1.9cm, clip]{Deliverables/first-deliverable/img/UseCase/UseCase-Table-Client.pdf}
\end{figure}

\newpage
\subsection{Use case table $-$ Produttore}
\begin{figure}[h!]
    \centering
    \includegraphics[trim= 1.8cm 8cm 5cm 1.9cm, clip]{Deliverables/first-deliverable/img/UseCase/UseCase-Table-Producer.pdf}
\end{figure}

\newpage
\subsection{Use case table $-$ Amministratore}
\begin{figure}[h!]
    \centering
    \includegraphics[trim= 1.8cm 13cm 5cm 1.9cm, clip]{Deliverables/first-deliverable/img/UseCase/UseCase-Table-Amministrator.pdf}
\end{figure}

\paragraph{Numerazione degli Use Case}

Gli \textbf{id} utilizzati nelle tabelle mostrate sopra sono stati così calcolati.

Prendiamo come esempio \textit{cercaProdotti}, \textit{modificaDatiAccount}, \textit{visualizzaProdotto} e \textit{visualizzaSegnalazioni} che si trovano nello Use Case diagram nella Figura \ref{fig:useCase}:



\begin{itemize}
    \item \textbf{cercaProdotti}: avrà come id 11 essendo che è il primo attore e che cercaProdotti è il primo use case per quell'attore.
    \item \textbf{modificaDatiAccount}: avrà come id 31 essendo che è il terzo attore e che modificaDatiAccount è il primo, e unico, use case per quell'attore
    \item \textbf{visualizzaProdotto}: avrà come id 551 essendo che è il quinto attore, il quinto use case di quell'attore e che è il primo use case associato a \textit{modificaProdotto}
    \item \textbf{visualizzaSegnalazioni}: avrà come id 63, stesso ragionamento per i precedenti, sesto attore e terzo use case associato all'attore.
\end{itemize}







\section{Diagrammi BPMN}

\newgeometry{
    margin=1.5cm,
    noheadfoot, nomarginpar,
    footskip=1.25em,
}
\begin{landscape}

\begin{figure}[h!]
    \centering
    \includegraphics[trim= 0cm 0cm 0cm 0cm, clip, width=0.95\linewidth]{Deliverables/first-deliverable/img/BPMN/BPMN-as-is-clientDomain.drawio.pdf}
    \caption{BPMN Cliente - dominio as-is }
\end{figure}

\end{landscape}

\newpage

\begin{landscape}

\begin{figure}[h!]
    \centering
    \includegraphics[trim= 0cm 0cm 0cm 0cm, clip, width=0.95\linewidth]{Deliverables/first-deliverable/img/BPMN/BPMN-lowLevel-createNewAccount.drawio.pdf}
    \caption{BPMN Produtore - Sign-Up Produtore }
\end{figure}

\end{landscape}

\restoregeometry




\end{document}
