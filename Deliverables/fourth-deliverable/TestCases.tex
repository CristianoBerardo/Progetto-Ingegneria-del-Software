\subsection{Test Case}

\begin{landscape}

\begin{longtable}
{p{0.4cm}|p{2.4cm}|p{0.4cm}|p{2.5cm}|p{3cm}|p{2.5cm}|p{2cm}|p{2.5cm}|p{2.5cm}|p{2cm}}
\caption{Test Cases Design Sprint \#2}\label{tab:TestCasesSprint2} \\
\hline
\textbf{ID} &\textbf{User Story} & \textbf{N.}& \textbf{Descrizione} & \textbf{Dati test case} & \textbf{Precondizioni} & \textbf{Dipendenze}& \textbf{Risultato atteso} & \textbf{Risultato effettivo} & \textbf{Note}  \\
\hline
\endfirsthead

\hline
\textbf{ID} &\textbf{User Story} & \textbf{N.} & \textbf{Descrizione} &\textbf{Dati test case} & \textbf{Precondizioni} & \textbf{Dipendenze}& \textbf{Risultato atteso} & \textbf{Risultato effettivo} & \textbf{Note}  \\
\hline
\endhead


\multirow{19}{0.2cm}{14} & \multirow{19}{0.2cm}{Inserimento prodotto} 
& 1 & Il produttore autenticato inserisce un prodotto & Vai al login del portale, inserisci: email e password, nella pagina clicca su l'icona + e inserisci tutti i dati richiesti & email e password devono esistere nel database& & Il nuovo prodotto deve essere presente lato database. Il produttore visualizza la lista aggiornata& Il prodotto viene inserito correttamente nel database& \\

&& 2 & Il produttore autenticato inserisce un prodotto e non fornisce i dati obbligatori richiesti & Vai al login del portale, inserisci: email e password, nella pagina clicca su l'icona + e non inserire i dati obbligatori  & email e password devono esistere nel database& & Il prodotto non viene inserito nel database e il produttore viene avvisato con un banner dell'errore& il bottone salva non viene attivato& \\

&& 3 & Il produttore autenticato inserisce un prodotto e inserisce un prezzo negativo & Vai al login del portale, inserisci: email e password, nella pagina clicca su l'icona + e inserisci un prezzo negativo & email e password devono esistere nel database& & Il prodotto non viene inserito nel database e il produttore viene avvisato con un banner dell'errore& il bottone salva non viene attivato& \\

&& 4 & Il produttore autenticato inserisce un prodotto e inserisce una quantità negativa & Vai al login del portale, inserisci: email e password, nella pagina clicca su l'icona + e inserisci un quantità negativa & email e password devono esistere nel database& & Il prodotto non viene inserito nel database e il produttore viene avvisato con un banner dell'errore& il bottone salva non viene attivato& \\

\hline
\hline
\newpage


\multirow{13}{0.2cm}{1} & \multirow{13}{0.2cm}{Registrazione cliente - produttore} 

& 5 & Il cliente può registrarsi alla piattaforma & Vai al login del portale, seleziona registrati come cliente, inserisci: username, email e password & l'email NON deve essere già presente nel database& & Il cliente viene registrato e viene inserito sia su MongoDB sia su Firebase& il nuovo cliente viene registrato sia lato mongoDB sia Firebase e gli viene assegnato il ruolo& \\

&& 6 & Il produttore può registrarsi alla piattaforma & Vai al login del portale, seleziona registrati come azienda, inserisci tutti i dati richiesti & l'email NON deve essere già presente nel database& & Il produttore viene registrato sia su MongoDB sia su Firebase& il nuovo produttore viene registrato sia lato mongoDB sia Firebase e gli viene assegnato il ruolo& \\

&& 7 & Il cliente si registra con una mail non valida& Vai al login del portale, seleziona registrati come cliente, inserisci una email errata (non compare @ e/o il dominio) &  & & Viene mostrato un banner di email non valida& Mostrato il messaggio di email non è valida& \\

&& 8 & Il produttore si registra con una mail non valida& Vai al login del portale, seleziona registrati come azienda, inserisci una email errata (non compare @ e/o il dominio) &  & & Viene mostrato un banner di email non valida& Mostrato il messaggio di email non è valida& \\

&& 9 & Il cliente si registra con una mail già esistente& Vai al login del portale, seleziona registrati come cliente, inserisci una email già presente&  & & Viene mostrato un banner di email precedentemente inserita& Mostrato il messaggio di email è già in uso& \\

&& 10 & Il produttore si registra con una mail già esistente& Vai al login del portale, seleziona registrati come azienda, inserisci una email già presente&  & & Viene mostrato un banner di email precedentemente inserita& Mostrato il messaggio di email è già in uso& \\

&& 11 & Il cliente si registra ma lascia vuoto uno o più campi obbligatori& Vai al login del portale, seleziona registrati come cliente, e lascia vuoto uno o più campi&  & & Viene mostrato un banner di errore& Mostrato un messaggio indicate il relativo campo mancante & \\

&& 12 & Il produttore si registra ma lascia vuoto uno o più campi obbligatori& Vai al login del portale, seleziona registrati come cliente, e lascia vuoto uno o più campi&  & & Viene mostrato un banner di errore& Mostrato un messaggio indicate il relativo campo mancante& \\

\hline
\hline
\newpage

\multirow{17}{0.2cm}{2} & \multirow{17}{0.2cm}{Autenticazione cliente - produttore} 

& 13 & Il cliente può autenticarsi sulla piattaforma & Vai al login del portale inserire email e password& Il cliente deve essersi precedentemente registrato& & Il cliente viene autenticato e riportato alla schermata home & Viene riportato nella schermata degli ordini&\\

&& 14 & Il produttore può autenticarsi sulla piattaforma & Vai al login del portale, seleziona registrati come azienda, inserisci tutti i dati richiesti & Il produttore deve essersi precedentemente registrato& & Il produttore viene autenticato e riportato alla sua schermata dedicata& Viene portato nella sua dashboard dedicata& \\

&& 15 & Il cliente tenta il login senza inserire l'email & Vai al login del portale, lascia vuoto il campo email, inserisci una password, premi 'Accedi'& Il cliente è registrato con una password nota&&  Il sistema mostra un messaggio di errore indicando che l'email è obbligatoria e non esegue l'autenticazione& Viene mostrato un messaggio di email non valida&\\

&& 16 & Il cliente tenta il login senza inserire la password & Vai al login del portale, inserisci un'email valida, lascia vuoto il campo password, premi 'Accedi'& Il cliente è registrato con un'email nota&&  Il sistema mostra un messaggio di errore indicando che la password è obbligatoria e non esegue l'autenticazione& Viene mostrato un messaggio di errore durante il login&\\

&& 17 & Il cliente tenta il login con un'email non valida (formato errato) & Vai al login del portale, inserisci un'email con formato non valido (non compare @ e/o il dominio), inserisci una password, premi 'Accedi'& && Il sistema mostra un messaggio di errore indicando che il formato dell'email non è valido e non esegue l'autenticazione & Viene mostrato un messaggio di email non valida&\\

&& 18 & Il cliente tenta il login con email valida ma password errata & Vai al login del portale, inserisci un'email registrata e una password errata, premi 'Accedi'& Il cliente è registrato con l'email inserita ma con una password differente&& Il sistema mostra un messaggio di errore e non esegue l'autenticazione& Viene mostrato un messaggio di errore durante il login&\\


&& 19 & Il cliente tenta il login con un'email non registrata & Vai al login del portale, inserisci un'email non presente nel sistema e una password qualsiasi, premi 'Accedi'& L'email inserita non corrisponde a nessun utente registrato&& Il sistema mostra un messaggio di errore e non esegue l'autenticazione&  Viene mostrato un messaggio di errore durante il login&\\


&& 20 & Il produttore tenta il login senza inserire l'email & Vai al login del portale (sezione azienda), lascia vuoto il campo email, inserisci una password, premi 'Accedi'& Il produttore è registrato con una password nota&&  Il sistema mostra un messaggio di errore indicando che l'email è obbligatoria e non esegue l'autenticazione& Viene mostrato un messaggio di email non valida&\\


&& 21 & Il produttore tenta il login senza inserire la password & Vai al login del portale (sezione azienda), inserisci un'email valida, lascia vuoto il campo password, premi 'Accedi'& Il produttore è registrato con un'email nota&&  Il sistema mostra un messaggio di errore indicando che la password è obbligatoria e non esegue l'autenticazione& Viene mostrato un messaggio di errore durante il login&\\

&& 22 & Il produttore tenta il login con un'email non valida (formato errato) & Vai al login del portale (sezione azienda), inserisci un'email con formato non valido (non compare @ e/o il dominio), inserisci una password, premi 'Accedi'& Nessuna specifica, il controllo è sul formato&&  Il sistema mostra un messaggio di errore indicando che il formato dell'email non è valido e non esegue l'autenticazione& Viene mostrato un messaggio di email non valida&\\

&& 23 & Il produttore tenta il login con email valida ma password errata & Vai al login del portale (sezione azienda), inserisci un'email registrata e una password errata, premi 'Accedi'& Il produttore è registrato con l'email inserita ma con una password differente&&  Il sistema mostra un messaggio di errore "Email o password non validi" (o simile) e non esegue l'autenticazione& Viene mostrato un messaggio di errore durante il login&\\

&& 24 & Il produttore tenta il login con un'email non registrata & Vai al login del portale (sezione azienda), inserisci un'email non presente nel sistema e una password qualsiasi, premi 'Accedi'& L'email inserita non corrisponde a nessun produttore registrato&&  Il sistema mostra un messaggio di errore "Email o password non validi" (o simile, per non rivelare se l'email esiste) e non esegue l'autenticazione& Viene mostrato un messaggio di errore durante il login&\\

\hline
\hline

\newpage
\multirow{6}{0.2cm}{14} & \multirow{6}{0.2cm}{Inserimento prodotto} 

& 25 & Eliminazione di un prodotto & Con Jest inviare una richiesta di Delete di un prodotto inserendo un identificativo & L'identificativo deve essere presente nel database& & Messaggio di successo da parte del endpoint della API e eliminazione lato database & Test superato & \\

&& 26 & Eliminazione di un prodotto inesistente & Con Jest inviare una richiesta di Delete di un prodotto inserendo un identificativo & L'identificativo non deve essere presente nel database& & Messaggio di errore da parte del endpoint della API & Test superato & \\

\hline
\hline
\newpage

\multirow{7}{0.2cm}{20} & \multirow{7}{0.2cm}{Aggiornamento prodotto} 

& 27 & Inserimento di un prodotto & Con Jest inviare una richiesta di creazione di un prodotto inserendo tutti i campi necessari& L'identificativo del produttore deve essere già presente nel database  &&  Messaggio di successo da parte del endpoint della API e creazione del prodotto lato database& Test superato & \\

&& 28 & Inserimento di un prodotto prodotto lasciando vuoti i campi obbligatori & Aprire Jest e inviare una richiesta di creazione di un prodotto lasciando vuoti i campi obbligatori (es. nome produttore) &&&Messaggio di errore da parte del endpoint della API & Test superato & \\

\hline
\hline

\multirow{3}{0.2cm}{5} & \multirow{3}{0.2cm}{Ricerca prodotti} 

& 29 & Ricerca prodotti & Da interfaccia grafica utilizzare i filtri messi a disposizione per la ricerca &&&  Vedere se i dati a schermo sono stati filtrati correttamente& Test superato & \\

\hline
\hline

\multirow{2}{0.2cm}{6} & \multirow{2}{0.2cm}{Ricerca produttore} 

& 30 & Ricerca produttore & Da interfaccia grafica utilizzare cerca produttore &&&  Vedere se i dati a schermo sono stati filtrati correttamente& Test superato & \\

\hline
\hline


\end{longtable}


\end{landscape}
\restoregeometry


\subsection{Sprint Review}

Durante la Sprint Review del secondo sprint del progetto Agritrento, sono state presentate e valutate le funzionalità sviluppate rispetto agli obiettivi definiti nello Sprint Planning.

Funzionalità completate e mostrate:
\begin{itemize}
    \item Implementazione di una migliore UI
    \item Gestione dei ruoli lato firebase e mongoDB
    \item Ricerca prodotti e produttori
    \item Creazione di un carrello
    \item Creazione di un ordine composto da più prodotti, eseguibile solo da un cliente
    \item Dashboard lato produttore dove può aggiornare e vedere la disponibilità dei propri prodotti
    \item Dashboard lato amministratore, solo UI, dove può vedere delle segnalazioni, le notifiche e gestire i produttori
    \item Implementazione di endpoint autenticati (Rimuovi prodotto, crea prodotto, rimuovi ordine, crea ordine,  ecc. lato cliente e produttore) dove viene verificato che chi manda il token JWT abbia un ruolo coerente con l'azione che vuole eseguire.
    \item Inserimento delle immagini per i prodotti più comuni
    \item Recupero della password di un utente
\end{itemize}

Il team ha riconfermato la stabilità delle vecchie API V1 e già alcune sono state sostituite con la V2 che richiede l'autenticazione. 

I prossimi sviluppi sarebbero quelli di :
\begin{itemize}
    \item Introdurre la paginazione, già presente ma non funzionante ancora.
    \item Aggiungere nuovamente le quantità dei prodotto nel caso un cliente eliminasse l'ordine
    \item Introdurre un metodo di pagamento in app
    \item Iniziare a far conoscere l'applicazione ai produttori locali
    \item Introdurre nella dashboard produttore la possibilità di accettare o rifiutare l'ordine
\end{itemize}

\vspace{0.5cm}

\subsection*{Product Backlog Refinement}
Il meeting di raffinamento ha permesso di riallineare le priorità e ottimizzare la pianificazione in base agli sviluppi dello Sprint 2.


\textbf{Incontro di raffinamento backlog:}

Durante il Product Backlog Refinement il team non ha trovato User Story da eliminare o inserire.

\vspace{0.5cm}

\subsection{Sprint Retrospective}

\textbf{Conclusioni e osservazioni:}
\begin{itemize}
    \item \textbf{Cosa ha funzionato:} In generale c'è stata una collaborazione da parte di tutti i membri del team.
    
    \item \textbf{Cosa non ha funzionato:} Ribadiamo quanto scritto nello sprint precedente, ovvero di una documentazione a tratti molto pesante e noiosa. La gestione del tempo per via degli esami prima della deadline del D4 ha rallentato un po' il progetto, ma dato che volevamo presentare l'applicazione il più completa possibile abbiamo continuato a lavorare intensamente nonostante le tempistiche ridotte.
    
    \item \textbf{Adozione di pratiche Agile:} Ci è capitato di ritrovarci per programmare tutti assieme e soprattutto per avere consigli dagli altri membri del team.
    
    \item \textbf{Dinamiche di gruppo:} abbiamo notato ancora una buona disponibilità del team, nonostante i vari impegni per quanto riguarda tirocini, tesi, lavoro, studio, ecc.
    
    \item \textbf{Spunti per il prossimo Sprint:}
    \begin{itemize}
        \item Implementazione pagamento.
        \item Implementazione delle notifiche, segnalazioni e valutazione del produttore da parte del cliente.
        \item Rendere utilizzabile la dashboard del amministratore.
    \end{itemize}
\end{itemize}